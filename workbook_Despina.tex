
\documentclass[openany,draft]{article}
\usepackage[utf8]{inputenc}
\usepackage[bulgarian]{babel}


%\use

\date{}

\title{Книжка за упражнителни задачки на Деспина}
\begin{document}
	
	
	\maketitle
	
	
	\section{Kвадратни уравнения и системи}
	
		
	\begin{enumerate}
		\item системи уравнения
		\item квадратни уравнения
		\item неравенства (???)
		\item други уравнения
	\end{enumerate}
	
	
	
	Фромули, които се изпозлват за квадратни уравния: \\
	Ако е дадено уравнение $ax^2 + bx + c = 0  $, имаме дискриминанта  $ D= b^2 - 4ac $, тогава решенията се задават с $ x_{1,2} = \frac{-b \pm \sqrt{D}  }{2a}$. Да разгледаме еднин пример. 
	Упражение(?): $(x - \frac{-b + \sqrt{D}  }{2a}  )(x - \frac{-b - \sqrt{D}  }{2a}) = ax^2 + bx +c $

	
	Припомняме формулите за съкратено умножение:\\
	 $(a+b)^2 = a^2 + 2ab + b^2$ \\
	 $(a-b)^2 = a^2 - 2ab + b^2$ \\
	 $(a+b)(a-b) = a^2 - b^2 $
	 
	 \vspace{1cm}
	 
	 
	
	Упражнителни задачи, които Деспина е решавала сама: \\
	
		$$x^2 - 5x + 6 = 0$$  \\
		
		
		\vspace{3cm}
		
		Още примери за решаване: \\
		\begin{enumerate}
			\item $x^2 - 6x + 8 = 0$
			\item $x^2 - 5x + 6 = 0$
			\item $x^2 - 5x + 6 = 0$
			\item $x^2 - 5x + 6 = 0$
			\item $x^2 - 5x + 6 = 0$
			\item $x^2 - 5x + 6 = 0$
		\end{enumerate}

		
		
		\newpage
	\section{Еднаквост и подобност на триъгълници}	
	
	Важно! Един триъгълник се определя от "три неща" - 
	три страни, две страни и ъгъл между тях, страна и два ъгъла. \\
	
	
	Признаци за еднаквост:
	\begin{enumerate}
		\item две страни и ъгъл между тях = две страни и ъгъл между тях $=>$  еднакви
		\item страна и два ъгъла = страна и два ъгъла $=>$  еднакви
		\item три страни = три страни $=>$ $ $ еднакви
	\end{enumerate}

	\vspace{1cm}
	
	
	Важно! Подобните триъгълници си приличат по това, че имат една и съща форма, но единият е 10 пъти или 5 пъти(или колкото и да е пъти) "по-голям" от другия \\
	
		Признаци за подобност:(Трябва да се потвърди от учебник)
	\begin{enumerate}
		\item (???) две страни са 5 пъти по-малки и ъгълът между тях е равен.
		\item (???) една страна е 5 пъти по-малка и 2 ъгъла са равни.
 		\item  (???) трите ъгъла са равни
 	\end{enumerate}
	
	
	
	\vspace{2cm}
	ирационални изрази, прогресии, статистика и обработка на данни, 
	решаване на триъгълник- sin, cos, tg, cotg в (0,180), синусова и косинусова теорема (?), елементи от стереометрията
	
		

	
\end{document}	
	
