
\documentclass{article}
\usepackage[utf8]{inputenc}
\usepackage[bulgarian]{babel}



\usepackage{cmap}
\usepackage[utf8]{inputenc}
\usepackage[T2A]{fontenc}

%\use
\newtheorem{problem}{Задача}

\newcounter{solution}

\newcommand\solution{%
	\stepcounter{solution}%
	\textbf{Решение :}\\%
}


\date{}

\title{Книжка за упражнителни задачки на Деспина}
\begin{document}
	
	
	\maketitle
	
	
	\section{Теория}
	
	
	
	\section{Kвадратни уравнения и системи}
	
		
	\begin{enumerate}
		\item системи уравнения
		\item квадратни уравнения
		\item неравенства (???)
		\item други уравнения
	\end{enumerate}
	
	
	
	Фромули, които се изпозлват за квадратни уравния: \\
	Ако е дадено уравнение $ax^2 + bx + c = 0  $, имаме дискриминанта  $ D= b^2 - 4ac $, тогава решенията се задават с $ x_{1,2} = \frac{-b \pm \sqrt{D}  }{2a}$. Да разгледаме еднин пример. 
	Упражение(?): $(x - \frac{-b + \sqrt{D}  }{2a}  )(x - \frac{-b - \sqrt{D}  }{2a}) = ax^2 + bx +c $

	
	Припомняме формулите за съкратено умножение:\\
	 $(a+b)^2 = a^2 + 2ab + b^2$ \\
	 $(a-b)^2 = a^2 - 2ab + b^2$ \\
	 $(a+b)(a-b) = a^2 - b^2 $
	 
	 \vspace{1cm}
	 
	 
	
	Упражнителни задачи, които Деспина е решавала сама: \\
	
		$$x^2 - 5x + 6 = 0$$  \\
		
		
		\vspace{3cm}
		
		Още примери за решаване: \\
		\begin{enumerate}
			\item $x^2 - 6x + 8 = 0$
			\item $x^2 - 5x + 6 = 0$
			\item $x^2 - 5x + 6 = 0$
			\item $x^2 - 5x + 6 = 0$
			\item $x^2 - 5x + 6 = 0$
			\item $x^2 - 5x + 6 = 0$
		\end{enumerate}

		
		
		\newpage
	\section{Еднаквост и подобност на триъгълници}	
	
	Важно! Един триъгълник се определя от "три неща" - 
	три страни, две страни и ъгъл между тях, страна и два ъгъла. \\
	
	
	Признаци за еднаквост:
	\begin{enumerate}
		\item две страни и ъгъл между тях = две страни и ъгъл между тях $=>$  еднакви
		\item страна и два ъгъла = страна и два ъгъла $=>$  еднакви
		\item три страни = три страни $=>$ $ $ еднакви
	\end{enumerate}

	\vspace{1cm}
	
	
	Важно! Подобните триъгълници си приличат по това, че имат една и съща форма, но единият е 10 пъти или 5 пъти(или колкото и да е пъти) "по-голям" от другия \\
	
		Признаци за подобност:(Трябва да се потвърди от учебник)
	\begin{enumerate}
		\item (???) две страни са 5 пъти по-малки и ъгълът между тях е равен.
		\item (???) една страна е 5 пъти по-малка и 2 ъгъла са равни.
 		\item  (???) трите ъгъла са равни
 	\end{enumerate}
	
	
	
	\vspace{2cm}
	ирационални изрази, прогресии, статистика и обработка на данни, 
	решаване на триъгълник- sin, cos, tg, cotg в (0,180), синусова и косинусова теорема (?), елементи от стереометрията
	
		


\section{Задачи с текс}

\subsection{Линейни уравнения и неравенства}

\begin{problem}
	В един магазин продали 488 кг портокали, лимони и маслини. Портокалите били с 40 кг повече от лимоните, а маслините - 5 пъти по-малко от портокалите. По колко килограма са продали от всеки вид?
\end{problem}

\begin{problem}
	През един сезон в консервната фабрика "Добруджанка" са обработили по 48 т домати на ден. След като предали 1300 т пресметнали, че това е с 524тт по-малко от цялото количество домати. Колко дни въъв фабриката са обработвани домати?
\end{problem}


\begin{problem}
	Обиколката на един триъгълник е 126 см. Едната му страна е с 12 см по-къса от другата , а третатат е 3/ от сбора на првите две. Да се намери най--голямата страна на този триъгълник.
\end{problem}


\begin{problem}
	Попитали Николай на колко е години, а той отговорил: "Мама е на 38 години. Тя е с 2 години по-млада от татко. Татко пък има два пъти повче години, отколкото аз и сестра ми заедно. Но аз със с 4 години по-малък от сестра ми." На коолко години са Николай и сестра му?
\end{problem}


\begin{problem}
	Един работник може да свърши определена работа за 15 дни, а друг работник за същото време свършва само 75 \% от тази работа. Отначало ддвамата  работници работели заедно 6 дни, а след това вторият само довършил останалата част. За колко дни била свършена цялата работа и какъв процент от нея е изработил всеки един работник?
\end{problem}

\subsection{Басейни}

\begin{problem}
	Един басейн се пълни от една тръба за 2 ч, от друга за 3ч, от трета за 4ч. За колко време се пълни от трите едновременно?
\end{problem}


\begin{problem}
	Един басейн се пълни от една тръба за 2 ч, от друга за 3ч. За колко време се пълни от двете едновременно?
\end{problem}

\solution









	
\end{document}	
	
