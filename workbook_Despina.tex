
\documentclass{article}
\usepackage[utf8]{inputenc}
\usepackage[bulgarian]{babel}

\usepackage{systeme}
\usepackage{amsmath}

\usepackage{cmap}
\usepackage[utf8]{inputenc}
\usepackage[T2A]{fontenc}

\newtheorem{definition}{Дефиниция}

\usepackage{comment}
%\use
\newtheorem{problem}{Задача}

\newcounter{solution}

\usepackage{graphicx}

\newcommand\solution{%
	\stepcounter{solution}%
	\textbf{Решение :}\\%
}


\date{}

\title{Книжка за упражнителни задачки на Деспина}
\begin{document}
	
	
	\maketitle
	
	
	\section{Теория}
	\includegraphics{Trig1}
	
	\vspace{-8cm}
	
	\begin{definition}$ sin(\alpha) = \frac{a}{c} $, $ cos(\alpha) = \frac{b}{c} $, $tg(\alpha) = \frac{a}{b} $, $cotg(\alpha) = \frac{b}{a} $	
	\end{definition}
Да зебележим, че $ sin(\beta) = cos(\alpha) = \frac{b}{c} $ и аналогично $cos(\beta) = sin(\alpha ) = \frac{a}{c} $.
$a^2 + b^2 = c^2  \to (\frac{a}{c})^2 + (\frac{b}{c})^2 = 1 \to sin^2(\alpha) + cos^2(\alpha) = 1$. \\
Тригонометрични тъждества: \\
$sin^2(\alpha) + cos^2(\alpha) = 1$ \\
$ sin(\alpha) = cos(\beta) = cos(90 - \alpha) $ \\
$ tg(\alpha)cotg(\alpha) = 1 $

	\section{Входно ниво 10ти клас}
	
	\section{Kвадратни уравнения и системи}
	
		
	\begin{enumerate}
		\item системи уравнения
		\item квадратни уравнения
		\item неравенства (???)
		\item други уравнения
	\end{enumerate}
	
	
	
	Фромули, които се изпозлват за квадратни уравния: \\
	Ако е дадено уравнение $ax^2 + bx + c = 0  $, имаме дискриминанта  $ D= b^2 - 4ac $, тогава решенията се задават с $ x_{1,2} = \frac{-b \pm \sqrt{D}  }{2a}$. Да разгледаме еднин пример. 
	Упражение(?): $(x - \frac{-b + \sqrt{D}  }{2a}  )(x - \frac{-b - \sqrt{D}  }{2a}) = ax^2 + bx +c $

	
	Припомняме формулите за съкратено умножение:\\
	 $(a+b)^2 = a^2 + 2ab + b^2$ \\
	 $(a-b)^2 = a^2 - 2ab + b^2$ \\
	 $(a+b)(a-b) = a^2 - b^2 $
	 
	 \vspace{1cm}
	 
	 
	
	Упражнителни задачи, които Деспина е решавала сама: \\
	
		$$x^2 - 5x + 6 = 0$$  \\
		
		
		\vspace{3cm}
		
		Още примери за решаване: \\
		\begin{enumerate}
			\item $x^2 - 6x + 8 = 0$
			\item $x^2 - 5x + 6 = 0$
			\item $x^2 - 5x + 6 = 0$
			\item $x^2 - 5x + 6 = 0$
			\item $x^2 - 5x + 6 = 0$
			\item $x^2 - 5x + 6 = 0$
		\end{enumerate}

		
		
		\newpage
	\section{Еднаквост и подобност на триъгълници}	
	
	Важно! Един триъгълник се определя от "три неща" - 
	три страни, две страни и ъгъл между тях, страна и два ъгъла. \\
	
	
	Признаци за еднаквост:
	\begin{enumerate}
		\item две страни и ъгъл между тях = две страни и ъгъл между тях $=>$  еднакви
		\item страна и два ъгъла = страна и два ъгъла $=>$  еднакви
		\item три страни = три страни $=>$ $ $ еднакви
	\end{enumerate}

	\vspace{1cm}
	
	
	Важно! Подобните триъгълници си приличат по това, че имат една и съща форма, но единият е 10 пъти или 5 пъти(или колкото и да е пъти) "по-голям" от другия \\
	
		Признаци за подобност:(Трябва да се потвърди от учебник)
	\begin{enumerate}
		\item (???) две страни са 5 пъти по-малки и ъгълът между тях е равен.
		\item (???) една страна е 5 пъти по-малка и 2 ъгъла са равни.
 		\item  (???) трите ъгъла са равни
 	\end{enumerate}
	
	
	
	\vspace{2cm}
	ирационални изрази, прогресии, статистика и обработка на данни, 
	решаване на триъгълник- sin, cos, tg, cotg в (0,180), синусова и косинусова теорема (?), елементи от стереометрията
	
		

\section{Тригонометрия}


\section{Задачи с текс}

\subsection{Линейни уравнения и неравенства}

\begin{problem}
	В един магазин продали 488 кг портокали, лимони и маслини. Портокалите били с 40 кг повече от лимоните, а маслините - 5 пъти по-малко от портокалите. По колко килограма са продали от всеки вид?
\end{problem}

\begin{problem}
	През един сезон в консервната фабрика "Добруджанка" са обработили по 48 т домати на ден. След като предали 1300 т пресметнали, че това е с 524тт по-малко от цялото количество домати. Колко дни въъв фабриката са обработвани домати?
\end{problem}


\begin{problem}
	Обиколката на един триъгълник е 126 см. Едната му страна е с 12 см по-къса от другата , а третатат е 3/ от сбора на првите две. Да се намери най--голямата страна на този триъгълник.
\end{problem}


\begin{problem}
	Попитали Николай на колко е години, а той отговорил: "Мама е на 38 години. Тя е с 2 години по-млада от татко. Татко пък има два пъти повче години, отколкото аз и сестра ми заедно. Но аз със с 4 години по-малък от сестра ми." На коолко години са Николай и сестра му?
\end{problem}


\begin{problem}
	Един работник може да свърши определена работа за 15 дни, а друг работник за същото време свършва само 75 \% от тази работа. Отначало ддвамата  работници работели заедно 6 дни, а след това вторият само довършил останалата част. За колко дни била свършена цялата работа и какъв процент от нея е изработил всеки един работник?
\end{problem}

\subsection{Басейни}

\begin{problem}
	Един басейн се пълни от една тръба за 2 ч, от друга за 3ч, от трета за 4ч. За колко време се пълни от трите едновременно?
\end{problem}


\begin{problem}
	Един басейн се пълни от една тръба за 2 ч, от друга за 3ч. За колко време се пълни от двете едновременно?
\end{problem}

\solution

\section{Системи}

\begin{problem}
	\[
	\begin{cases}
	x - y = 7 \\
	x^2 - xy - y^2=19 \hspace{2cm} x = y + 7 
	\end{cases}
	\]
\end{problem}

\noindent
\solution
$ x = y + 7 $ \\
$ (y+7)^2 - (y+7)y - y^2 = 19 $ \\
$ y^2 + 14y + 49 - y^2 - 7y - y^2 = 19 $ \\
$ -y^2 + 7y + 30 = 0 $ \\ 
$ y^2 - 7y - 30 = 0  \to a = 1, b = -7 , c = -30$ \\ 
$D = 49 + 120 = 169, y_1 = 10$ , $y_2 = -3 $ \\
$x_1 = 10 + 7 = 17, x_2 = -3 + 7 = 4$ \\
Отг. Решенията на системата са: $(17,10) , (4,-3)   $

\vspace{1cm}

\begin{problem}
	\[
	\begin{cases}
	2x - y -1 = 0 \\
	xy - 1 = 0  
	\end{cases}
	\]
\end{problem}
\solution
$ y = 2x -1$ \\
$ x(2x-1) - 1 = 0 $ \\
$ 2x^2 - x - 1 = 0 \to a=2, b = -1, c= -1$  \\
$ D = 1 - 4.2.(-1) = 9 $
$ x_1 = \frac{-(-1)+ \sqrt{9}}{2.2}= \frac{4}{4} = 1 $, $x_2 = \frac{-(-1)- \sqrt{9}}{2.2}= -\frac{2}{4} = -\frac{1}{2}  $
$ y_1 = 2x_1 - 1 = 2 - 1 = 1 $, $y_2 = 2x_2 - 1 = 2(-\frac{1}{2})-1 = -2  $ \\
Отг. $(1,1), (-\frac{1}{2}, -2  ) $

\begin{problem}
	\[
	\begin{cases}
	x + y = -2 \\
	x^2 + y^2 = 2 
	\end{cases}
	\]
\end{problem}

\begin{problem}
	\[
	\begin{cases}
	x - 3y + 1 = 0 \\
	x^2 - 4xy + 3y^2 + x - y = 0 
	\end{cases}
	\]
\end{problem}


\begin{comment}

\begin{align*}
x &= a^{\log_a (x)} \bigg\vert \ln () \\
\Leftrightarrow \qquad \ln (x) &= \ln \left(a^{\log_a (x)} \right) \\
\Leftrightarrow \qquad \ln (x) &= \log_a (x) \cdot \ln (a) \bigg\vert - \ln (a) \\
\Leftrightarrow \qquad \frac{\ln (x)}{\ln (a)} &= \log_a (x) 
\end{align*}

\begin{align*}
&\left\{
\begin{aligned}
I_{50}+I_{10}=I_{03}\\
I_{21}=I_{12}+I_{10}\\
I_{12}+I_{32}=I_{21}\\
I_{03}=I_{32}+I_{34}\\
I_{34}=I_{451}+I_{452}\\
\end{aligned}\right.
&
\left\{\begin{aligned}
U_{30}+U_{01}+U_{121}+U_{23}=0\\
U_{34}+U_{452}+U_{50}+U_{30}=0\\
U_{121}+U_{122}=0\\
U_{451}+U_{452}=0\\
\end{aligned}\right. \\
&\text{, pentru legea I, și}
&\text{pentru legea a II-a}
\end{align*}

\end{comment}



\end{document}	
	
	

